\documentclass[aspectratio=43,x11names]{beamer}
\usetheme{Pittsburgh}
\usepackage{xcolor}
\usepackage[utf8]{inputenc}
\usepackage[german]{babel}
\usepackage{amsmath}
\usepackage{amsfonts}
\usepackage{amssymb}
\usepackage{graphicx}
\usepackage{multicol}
\usepackage{wrapfig}
\usepackage{hyperref}

\author{Jonas Betzendahl}
\title{Machine Learning Science Slam}

\beamertemplatenavigationsymbolsempty 

% For Footnotes without markers on the slide
% https://tex.stackexchange.com/questions/30720/footnote-without-a-marker
\newcommand\blfootnote[1]{%
  \begingroup
  \renewcommand\thefootnote{}\footnote{#1}%
  \addtocounter{footnote}{-1}%
  \endgroup
}

\begin{document}

%------------------------------------------------------------------------------------
\section{Introduction}

\begin{frame}
\begin{center}
\Large \glqq Maschinelles Lernen\grqq
\normalsize 

(Und warum das vielleicht gruseliger ist, als es klingt)
\bigskip\bigskip

\Large Jonas Betzendahl\\
\texttt{@jbetzend}
\smallskip

\href{https://twitter.com/jbetzend}{\includegraphics[scale=0.125]{images/twitter_logo.png}}
\href{https://github.com/jbetzend}{\includegraphics[scale=0.125]{images/github_logo.png}}
\href{https://whispeer.de/en/user/jbetzend}{\includegraphics[scale=0.125]{images/whispeer_logo.png}}
\end{center}
\end{frame}

%------------------------------------------------------------------------------------

\begin{frame}
\frametitle{Wer bin ich?}

Jonas Betzendahl\\
Masterstudiengang \glqq Intelligente Systeme\grqq\\
Technische Fakultät, Universität Bielefeld\\
\texttt{jonas.betzendahl@gmail.com}
\smallskip\smallskip

\begin{center}
\includegraphics[scale=0.5]{images/thermopostkarte.jpg} 
\end{center}
\end{frame}

%------------------------------------------------------------------------------------

\begin{frame}
\frametitle{Small Talk in Intelligent Systems}

\begin{multicols}{2}

Die häufigste Frage an meinen Studiengang:

\pause

\begin{itemize}
\item \glqq Na, wie lange dauert es noch bis zur Roboterapokalypse?\grqq
\end{itemize}

\columnbreak

\includegraphics[height=0.7\textheight,keepaspectratio]{images/Robot-Apocalypse.jpg} 

\end{multicols}

\end{frame}

\begin{frame}
\begin{center}
\includegraphics[width=\textwidth]{images/amazon-logo.jpg} 
\end{center}
\end{frame}

\begin{frame}
\begin{center}
\includegraphics[width=\textwidth]{images/amazon-one-wallet.jpg} 
\end{center}
\end{frame}

\begin{frame}
\begin{center}
\includegraphics[width=\textwidth]{images/amazon-buncha-wallets.jpg} 
\end{center}
\end{frame}

%------------------------------------------------------------------------------------

\begin{frame}

\begin{center}
\dots zumindest habe ich bisher so immer meine Slams angefangen.
\pause\bigskip

Wir müssen reden!
\end{center}

\end{frame}

%------------------------------------------------------------------------------------

\section{Was ist Maschinelles Lernen?}

\begin{frame}
\begin{center}
\huge
Wie funktioniert\\Maschinelles Lernen?
\end{center}
\end{frame}

\begin{frame}
\begin{multicols}{2}
\includegraphics[scale=0.4]{images/artificial-intelligence-507813_640.jpg} 
\columnbreak

Maschinelles Lernen simuliert einen Vorgang nicht unähnlich dem im menschlichen Gehirn selbst.
\pause\bigskip

Ein (künstliches) \emph{neuronales Netz} wird simuliert und trainiert mit \emph{Testdaten}, bis es akzeptable Leistungen bringt.
\end{multicols}
\end{frame}

\begin{frame}
\begin{center}
\includegraphics[scale=0.375]{images/simple_neural_network_header.jpg} 
\end{center}
\end{frame}

%------------------------------------------------------------------------------------

\section{Die Vorteile von Maschinellem Lernen}

\begin{frame}
\begin{center}
\huge
Die Errungenschaften\\von Maschinellem Lernen
\end{center}
\end{frame}

\begin{frame}
Maschinelles Lernen ist prinzipiell sehr mächtig und nützlich\dots

\begin{center}
\includegraphics[scale=0.105]{images/autopilotnew.jpg} 
\end{center}
\end{frame}

\begin{frame}[fragile]
\small
\begin{verbatim}
 "Nachdem die Menschheit Jahrtausende damit verbracht
  hat, ihre Taktiken zu verbessern, erzählen uns die
  Computer, dass wir komplett daneben liegen.
  Ich würde soweit gehen, zu sagen, dass noch kein 
  einziger Mensch auch nur den Rand der Wahrheit von
  Go berührt hat."
                                             -- Ke Jie
\end{verbatim}

\begin{center}
\includegraphics[scale=0.65]{images/kejie.png} 
\end{center}
\end{frame}

%------------------------------------------------------------------------------------

\section{Die Probleme}

\begin{frame}
\begin{center}
\Large
Die Fehler von\\Maschinellem Lernen
\end{center}
\end{frame}

\begin{frame}

\begin{multicols}{2}

\includegraphics[scale=0.1]{images/sirifail.jpg} 

\columnbreak

ffoo
\end{multicols}
\end{frame}

\begin{frame}
\begin{center}
\includegraphics[height=0.65\textheight,keepaspectratio]{images/deep_neural_networks_1.png}
\end{center}
\end{frame}

\begin{frame}
\begin{center}
\includegraphics[height=0.65\textheight,keepaspectratio]{images/deep_neural_networks_2.png} 
\end{center}
\end{frame}

\begin{frame}
\begin{center}
\includegraphics[height=0.65\textheight,keepaspectratio]{images/deep_neural_networks_3.png} 
\end{center}
\end{frame}

\begin{frame}
\begin{center}
\includegraphics[height=0.65\textheight,keepaspectratio]{images/deep_neural_networks_4.png} 
\end{center}
\end{frame}

\begin{frame}
\begin{center}
\includegraphics[height=0.39\textwidth,keepaspectratio]{images/recog_panda.png} 
\end{center}
\end{frame}

\begin{frame}
\begin{center}
\includegraphics[height=0.39\textwidth,keepaspectratio]{images/recog_gibbon.png} 
\end{center}
\end{frame}

\begin{frame}
\frametitle{Wat lernt misch datt?}
\begin{center}
\color{red}
\large
Maschinelles Lernen liefert oft\\nur \emph{Ergebnisse}, keine \emph{Begründungen}.
\pause\bigskip

Außerdem ist das Ergebnis höchstens\\so allgemein wie die Trainingsdaten.
\end{center}
\end{frame}

%------------------------------------------------------------------------------------

\subsection{Adversarial Images}
\begin{frame}
\begin{center}
\huge
\glqq Adversarial Objects\grqq \\
\Large
(Feindliche Objekte)
\end{center}
\bigskip
\normalsize

(\textit{Subs., plural}) Objekte, die für das menschliche Auge herkömmlich erscheinen,
aber für den Computer radikal anders aussehen.
\end{frame}

\begin{frame}
\includegraphics[scale=0.5]{images/cat_adversarial.png} 
\end{frame}

\begin{frame}
\includegraphics[scale=0.5]{images/cat_rotated.png} 
\end{frame}

\begin{frame}

Feindliche 3D-gedruckte Schildkröte:
\bigskip

\includegraphics[scale=0.5]{images/rifle_turtle.jpg} 
\end{frame}

%------------------------------------------------------------------------------------

\begin{frame}
Ein paar offensichtliche Probleme:
\bigskip
\begin{multicols}{2}
\begin{itemize}
\pause\item Gratis T-Shirts die am Flughafen als Waffen erkannt werden
\pause\item Plakate neben der Autobahn die als Stoppschilder erkannt werden
\pause\item Lars
\item \dots
\end{itemize}
\columnbreak
\pause
\includegraphics[scale=0.5]{images/lars_eisbaer.png} 
\end{multicols}
\end{frame}

\begin{frame}
Ein paar offensichtliche Probleme:
\bigskip
\begin{multicols}{2}
\begin{itemize}
\item Gratis T-Shirts die am Flughafen als Waffen erkannt werden
\item Plakate neben der Autobahn die als Stoppschilder erkannt werden
\item Lars
\item \dots
\end{itemize}
\columnbreak
\includegraphics[scale=0.45]{images/lars_sweden.jpg} 
\end{multicols}
\end{frame}

\begin{frame}
Ein paar offensichtliche Probleme:
\bigskip
\begin{multicols}{2}
\begin{itemize}
\item Gratis T-Shirts die am Flughafen als Waffen erkannt werden
\item Plakate neben der Autobahn die als Stoppschilder erkannt werden
\item LARs
\item \dots
\end{itemize}
\columnbreak
\includegraphics[scale=0.275]{images/lars.jpg} 
\end{multicols}
\end{frame}

\begin{frame}
\frametitle{Wat lernt misch datt?}
\begin{center}
\color{red}
\large
Maschinelles Lernen ist nicht unfehlbar und darf in kritischen Systemen nie 
unüberprüft wichtige Entscheidungen treffen.
\end{center}
\end{frame}

%------------------------------------------------------------------------------------

\section{Menschliche Leichtgläubigkeit}

\begin{frame}
\frametitle{}
Fallbeispiel: \glqq Predictive Policing\grqq

\begin{center}
\includegraphics[scale=0.25]{images/predictive_policing.jpg} 
\end{center}
\end{frame}

\begin{frame}
\frametitle{Wat lernt misch datt?}
\begin{center}
\color{red}
\large
Maschinelles Lernen kann (potentiell) Vorurteile\\und Fehler
in der Datengrundlage und im Modell\\verstärken oder verschlimmern.
\end{center}
\end{frame}

%------------------------------------------------------------------------------------

\begin{frame}
\frametitle{Zusammenfassung}

Die guten Nachrichten:

\begin{itemize}
\color{Green4}
\pause\item Roboterapokalypse: Erstmal unwahrscheinlich
\pause\item Maschinelles Lernen kann uns das Leben sehr vereinfachen
\end{itemize}
\bigskip

Aber:

\begin{itemize}
\color{Firebrick1}
\pause\item Maschinelles Lernen ist nicht fehlerfrei
\pause\item Meist nur ein Ergebnis, keine Begründung
\pause\item Vorsicht vor Vorurteilen in der Datengrundlage
\pause\item Kritische Entscheidungen brauchen Menschen in der Schleife
\end{itemize}

\end{frame}

\end{document}