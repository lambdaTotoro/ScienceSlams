\documentclass[aspectratio=43,x11names]{beamer}
\usetheme{Pittsburgh}
\usepackage{xcolor}
\usepackage[utf8]{inputenc}
\usepackage[german]{babel}
\usepackage{amsmath}
\usepackage{amsfonts}
\usepackage{amssymb}
\usepackage{graphicx}
\usepackage{multicol}
\usepackage{wrapfig}
\usepackage{hyperref}
\usepackage{tikz}

\usetikzlibrary{shapes,arrows,chains}

\author{Jonas Betzendahl}
\title{Algorithmen und Du}

\beamertemplatenavigationsymbolsempty 

%src: https://tex.stackexchange.com/questions/34921/how-to-overlap-images-in-a-beamer-slide
\def\Put(#1,#2)#3{\leavevmode\makebox(0,0){\put(#1,#2){#3}}}

\begin{document}

%------------------------------------------------------------------------------------
\section{Introduction}

\begin{frame}
\begin{center}
\vfill
\huge Algorithmen und Du
\normalsize 
\smallskip
\smallskip

Warum Künstliche Intelligenz nie politisch neutral sein kann
\bigskip\bigskip

\large Jonas Betzendahl, M.Sc.
\bigskip\bigskip

\href{https://twitter.com/lambdatotoro}{\includegraphics[scale=0.125]{images/twitter_logo.png}}
\href{https://chaos.social/@lambdatotoro}{\includegraphics[scale=0.125]{images/mastodon_logo.png}}
\href{https://github.com/jbetzend}{\includegraphics[scale=0.125]{images/github_logo.png}}
\href{https://whispeer.de/en/user/jbetzend}{\includegraphics[scale=0.125]{images/whispeer_logo.png}}

\texttt{@lambdaTotoro (@chaos.social)}
\end{center}
\end{frame}

%------------------------------------------------------------------------------------


\begin{frame}
\begin{center}
\huge
Künstliche Intelligenz
\bigskip

\large
The hype is real\dots
\end{center}
\end{frame}

\begin{frame}
\frametitle{Menschen sind \emph{sehr} fehlbar...}
\begin{minipage}{0.45\textwidth}
\begin{center}
\includegraphics[keepaspectratio, height=0.7\textheight]{images/calipers_transparent}
\end{center}
\end{minipage}\begin{minipage}{0.45\textwidth}
\begin{center}
\pause
\includegraphics[keepaspectratio, height=0.7\textheight]{images/human_bias}
\end{center}
\end{minipage}
\end{frame}

\begin{frame}
\frametitle{...und \emph{sehr} langsam!}
\begin{center}
\includegraphics[height=0.6\textheight, keepaspectratio]{images/sloth}
\end{center}
\end{frame}

\begin{frame}
\begin{center}
\includegraphics[keepaspectratio, height=0.75\textheight]{images/elektronengehirn}
\end{center}
\pause
\Put(-10,400){\includegraphics[scale=0.4, angle=5]{images/ki_lufthansa}}
\pause
\Put(-20,200){\includegraphics[scale=0.25, angle=25]{images/gema_ki_klein}}
\pause
\Put(-10,350){\includegraphics[scale=0.4, angle=-15]{images/ki_suicide}}
\pause
\Put(-10,125){\includegraphics[scale=0.3, angle=-4]{images/ki_medicine}}
\pause
\Put(-10,250){\includegraphics[scale=0.35, angle=10]{images/ki_law}}
\pause
\Put(10,250){\includegraphics[scale=0.71]{images/spahn}}
\end{frame}

\begin{frame}
\frametitle{Bonusrunde!}
\begin{center}
\large
Science-Slam-ception!

\huge
\emph{Blockchains}
\pause\bigskip

\large
Macht eine Blockchain für mich, meine Firma\\ oder meine Organisation Sinn?
\pause
\bigskip\bigskip\bigskip

\Huge
\includegraphics[scale=0.65]{images/pointing.png} \textcolor{red}{\textbf{\textit{Nein!}}} \includegraphics[scale=0.65]{images/pointing2.png}
\end{center}
\end{frame}

%------------------------------------------------------------------------------------

\section{How it works and weaknesses}

\begin{frame}
\begin{center}
\huge
Algorithmen \&\\
Maschinelles Lernen
\bigskip

\large
Was ist das? Wie geht das?\\ Und was sind die Probleme?
\end{center}
\end{frame}

\begin{frame}
\frametitle{Algorithmen}
\begin{center}
\includegraphics[height=0.7\textheight, keepaspectratio]{images/recipe}

Alltäglicher als vielleicht gedacht\dots
\end{center}
\end{frame}


\begin{frame}
\frametitle{Maschinelles Lernen}
\begin{center}
\includegraphics[width=0.95\textwidth, keepaspectratio]{images/funnel}
\end{center}
\end{frame}

\begin{frame}[fragile]
\frametitle{Schicke Mathematik}

\begin{minipage}{0.52\textwidth}
\begin{center}
\includegraphics[scale=0.3]{images/linear_nonlinear.png} 
\end{center}
\end{minipage}\begin{minipage}{0.35\textwidth}
$$y_k = \varphi\left(\sum_{j=0}^m w_{kj}x_j\right)$$
\end{minipage}

\small
\begin{minipage}{0.47\textwidth}
$$\Delta w_{ij}=-\eta \frac{\partial E}{\partial w_{ij}} = -\eta o_i \delta_j$$
$$\lambda(x) = C(t-x)^2$$
$$r_i \dot{y_i} = -y_i \sum_{j=1}^n w_{ji}\sigma\left(y_j - \Theta_j\right) * I_i(t)$$
\end{minipage}\begin{minipage}{0.47\textwidth}
\begin{center}
\includegraphics[scale=0.35]{images/recursive.jpg} 
\end{center}
\end{minipage}
\end{frame}

\begin{frame}
\frametitle{Limitierungen}

Egal wie viele beeindruckende Formeln und tolle Buzzwords geworfen werden, es gilt immer\dots\bigskip

\begin{itemize}
\pause\item Wir können nur lernen, was schon in den Daten drin ist!\\ Auf welchen Daten lernen wir?\\

\pause\item Die Maschine optimiert, weiß aber nicht was \glqq optimal\grqq\ bedeutet! Wer bestimmt worauf optimiert sind und was hat das für Folgen?\\

\pause\item Wenn KI und Gesellschaft aufeinander treffen besteht die Gefahr der selbsterfüllenden Prophezeihung!
\end{itemize}
\end{frame}

%------------------------------------------------------------------------------------

\section{Surely nobody would be this stupid?}

\begin{frame}
\begin{center}
\huge
KI in der freien Wildbahn
\bigskip

\large
Leider nicht nur theoretisch problematisch\dots
\end{center}
\end{frame}

\begin{frame}
\frametitle{Erkennung von Panzern}
\begin{center}
\includegraphics[height=0.7\textheight, keepaspectratio]{images/tank}

Der vielleicht spektakulärste Fehlschlag des Feldes!
\pause
\Put(-250,250){\includegraphics[scale=0.45, angle=15]{images/princess_bride.jpg} }
\end{center}
\end{frame}

\begin{frame}
\frametitle{Toxic Discourse for Fun and Profit}
\begin{center}
\includegraphics[width=0.9\textwidth, keepaspectratio]{images/youtube}
\bigskip

$1.000.000.000$ Stunden pro Tag\\
Ziel erreicht, Verantwortung verfehlt!
\pause
\Put(-220,250){\includegraphics[scale=0.2, angle=-15]{images/goldblum} }
\end{center}
\end{frame}

\begin{frame}
\frametitle{Bewertung von Chancen am Arbeitsmarkt?}
\begin{minipage}{.6\textwidth}
\includegraphics[width=0.8\textwidth, keepaspectratio]{images/negativfaktor}
\end{minipage}\hspace*{-25pt}\begin{minipage}{.4\textwidth}
Das \texttt{AMS-Arbeitsmarkt} \texttt{-Chancen-Modell}, die zukünftige Basis für Arbeitslosenunterstützung in Österreich, rechnet allen Frauen einen negativen Faktor an.
\end{minipage}

\pause
\Put(50,250){\includegraphics[scale=1.8, angle=0]{images/any_of_this} }
\end{frame}

\begin{frame}
\frametitle{Erkennen von sexueller Orientierung?}
\begin{minipage}{.6\textwidth}
\includegraphics[width=0.8\textwidth, keepaspectratio]{images/paper_orientation}
\end{minipage}\hspace*{-25pt}\begin{minipage}{.4\textwidth}
\includegraphics[width=0.99\textwidth, keepaspectratio]{images/composite_faces}
\end{minipage}

\pause
\Put(0,250){\includegraphics[scale=0.45, angle=15]{images/princess_bride.jpg} }
\pause
\Put(0,250){\includegraphics[scale=0.2, angle=-15]{images/goldblum} }
\end{frame}

\begin{frame}
\frametitle{Erkennen von Maskierten?}
\begin{minipage}{.6\textwidth}
\includegraphics[width=0.8\textwidth, keepaspectratio]{images/paper_disguised}
\end{minipage}\hspace*{-25pt}\begin{minipage}{.6\textwidth}
\includegraphics[width=0.8\textwidth, keepaspectratio]{images/disguised}
\end{minipage}

\pause
\Put(0,250){\includegraphics[scale=0.2, angle=-15]{images/goldblum} }
\pause
\Put(50,250){\includegraphics[scale=1.8, angle=0]{images/any_of_this} }
\end{frame}

\begin{frame}
\frametitle{``Phrenology as a Service''}
\begin{center}
\includegraphics[keepaspectratio, height=0.7\textheight]{images/hirevue}

\large
Gesichtsanalyse im Bewerbungsgespräch
\end{center}
\pause
\Put(90,305){\includegraphics[scale=0.3]{images/calipers_transparent}}
\end{frame}

\begin{frame}
\begin{center}
\includegraphics[keepaspectratio, height=0.85\textheight]{images/thomas}
\end{center}
\end{frame}

%------------------------------------------------------------------------------------

\section{What to do about it?}

\begin{frame}
\begin{center}
\huge
Was tun wir dagegen?

\large
take-home messages
\normalsize
\end{center}
\bigskip

Jedes Mal wenn \glqq Künstliche Intelligenz\grqq\ oder \glqq Maschinelles Lernen\grqq\ aufkommen:
\medskip

\begin{itemize}
\pause\item \emph{Woher} wird gelernt? Welches \emph{Datenset} wird benutzt?
\pause\item \emph{Was} wird gelernt? Worauf wird \emph{optimiert}?
\pause\item Was sind die \emph{Folgen} von diesen Entscheidungen?
\end{itemize}

\pause\bigskip\bigskip
\begin{center}
\large
skeptisch bleiben!
\end{center}
\Put(235,45){\includegraphics[scale=0.5, keepaspectratio]{images/thinking_emoji}}
\end{frame}

%------------------------------------------------------------------------------------

\begin{frame}
\frametitle{Quellen:}
\scriptsize
\begin{itemize}
\item Diskriminierung b. Wohnungssuche ``Hanna und Ismail'', Köppen et al.  2017\\
\url{https://www.hanna-und-ismail.de/}

\item ``Dynamic fairness - Breaking vicious cycles in automatic decision making'',
Paaßen et al., 2019\\ \url{https://arxiv.org/abs/1902.00375}

\item ``Disguised Face Identification (DFI) with Facial KeyPoints using Spatial Fusion Convolutional Network'', Singh et al., 2017\\ \url{https://arxiv.org/abs/1708.09317}

\item ``Deep Neural Networks Are More Accurate Than Humans at Detecting Sexual Orientation From Facial Images'', Kosinski et al., 2018\\ \url{https://www.gsb.stanford.edu/faculty-research/publications/deep-neural-networks-are-more-accurate-humans-detecting-sexual}

\item ``Negativfaktor Frau'', Ingrid Brodnig, 2018 \\ \url{https://www.profil.at/meinung/ingrid-brodnig-negativfaktor-frau-10422551}

\item ``AI used for first time in job interviews in UK to find best applicants'', The Telegraph, 2019-09-27 \\ \url{https://www.telegraph.co.uk/news/2019/09/27/ai-facial-recognition-used-first-time-job-interviews-uk-find/}

\end{itemize}
\end{frame}
\end{document}

